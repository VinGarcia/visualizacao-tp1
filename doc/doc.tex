% -----------------------------------------------------------------------------
% Document type
\documentclass[12pt]{article}

% -----------------------------------------------------------------------------
% Packages
\usepackage{amsmath}
\usepackage{amssymb}
\usepackage{graphicx}
\usepackage{sbc-template}
\usepackage[utf8]{inputenc}
\usepackage[pdftex]{hyperref}
\usepackage[brazilian]{babel}
\usepackage[lined,ruled,linesnumbered]{algorithm2e}

% -----------------------------------------------------------------------------
% Definitions
\title{Trabalho Prático 1 \\ Visualização de Padrões em Comunidades}
\author{Larissa Leijôto, Péricles Alves, Rubens Emilio, Vinícius Garcia}
\address{Departamento de Ciência da Computação \\
	Universidade Federal de Minas Gerais
	\email{$[$leijoto,periclesrafael,rubens,garcia$]$@dcc.ufmg.br}
}

\begin{document}
\hypersetup{pdfborder = {0 0 0}}
\maketitle

% -----------------------------------------------------------------------------
% Introdução
\section{Introdução}
Visualizar informação não é uma tarefa simples: além dos problemas inerentes
ao processamento dos dados, construir visualizações claras e objetivas requer
conhecimentos básicos sobre o processo cognitivo.
Deixando de lado as etapas de coleta e tratamento da base de dados --
que são, em si, problemas complexos --, existem diversas estratégias para
extrair informações relevantes de conjuntos de dados.

Diversos algoritmos de áreas como mineração de dados, aprendizado de máquina
e computação natural, processam os dados a fim de identificar padrões
implícitos. O volume dos dados e a existência de ruídos são exemplos
de problemas recorrentes nestas áreas de pesquisa. Ao final do processamento,
é necessário exibir as informações de forma objetiva, valorizando os resultados
obtidos, e permitindo novas conclusões sobre os dados.

A tarefa de visualizar dados, além de lidar com os desafios discutidos
acima, ainda depende de conhecimentos acerca do processo cognitivo.
É preciso ter um mínimo de noção sobre como o ser humano percebe e interpreta
informações visuaos. Só então é possível decidir como exibir resultados
pré-computados, ou ainda, como exibir os dados iniciais para que o leitor
consiga, com relativamente pouco esforço, tirar conclusões valiosas.

Neste trabalho, será implementada uma técnica de visualização de padrões
em grafos. A visualização será feita sobre conjuntos disjuntos de pessoas
(comunidades), que compartilham um conjunto predefinido de lugares
que costumam visitar. O resultado do trabalho é uma representação gráfica
para cada conjunto de pessoas. Assim, espera-se que, ao analisar mais de uma
comunidade, seja possível identificar padrões visuais entre as mesmas.

% -----------------------------------------------------------------------------
% Padrões em Comunidades 
\section{Padrões em Comunidades}
Os dados a serem analisados contêm comunidades compostas por pessoas,
que, por sua vez, podem visitar diferentes lugares. O objetivo principal
de representar tais comunidades visualmente é conseguir identificar,
com pouco esforço, correlações entre pessoas de comunidades distintas.

\subsection{Donut Graphs}
Para representar cada indivíduo, será utilizada uma técnica de representação
do tipo gráfico de pizza, denominada {\it donut graph}. Este tipo de gráfico 
apresenta, de forma bastante clara, a frequência com que determinado indivíduo
visita diferentes lugares.

Além da espessura de cada fatia do gráfico indicar a frequência de visitas,
é possível associar cores a cada lugar. Isto possibilita a identificação
de padrões visuais, principalmente quando diferentes gráficos (indivíduos)
são comparados.

\subsection{Posicionamento e Arestas}
O posicionamento de cada pessoa em sua respectiva comunidade é utilizado
para representar a semelhança entre pessoas de diferentes comunidades.
Desta forma, se uma pessoa tem hábitos similares aos de uma pessoa de outra
comunidade, ou seja, se visitam os mesmos tipos de lugares, então existe
uma correlação entre elas, e o posicionamento das mesmas no espaço será igual.

Além disso, sempre que pessoas de uma mesma comunidade visitam um mesmo lugar,
é desenhada uma aresta entre as mesmas, facilitando a identificação
de mais padrões visuais. Neste caso, os padrões serão referentes às conexões
entre cada {\it donut graph}.

\subsection{* Heatmap}
A fim de explorar outras técnicas de visualização, procuramos alguma
abordagem que sumarizasse as correlações entre as pessoas de diferentes
comunidades. Com isto, decidimos implementar o modelo de visualização
por {\it heatmap}, muito úteis na representação de grandes quantidades
de dados, e sem o problema de sobrepor informações.

No caso da análise de similaridade entre comunidades, criamos uma representação
para cada indivíduo. Uma pessoa é descrita como um vetor da frequência com
que a mesma visita lugares de tipos diferentes. Aos valores de frequência
são associadas diferentes gradações de cores. 

% -----------------------------------------------------------------------------
% Desafios e Detalhes de Implementação
\section{Desafios e Detalhes de Implementação}
Após a implementação das visualizações, foi possível tirar algumas conclusões
a respeito dos dados analisados. Tais conclusões são obtidas a partir
de padrões visuais nas imagens criadas.

\subsection{Posicionamento de indivíduos nos gráficos de tipo Donut}
Ao plotar os plotar os gráficos de tipo {\it donut}, foi preciso definir
posições padrão para cada indivíduo. A fim de minimizar a poluição visual,
decidimos posicionar os indivíduos dispondo-os sobre um círculo.
Desta forma, tanto conseguimos evitar sobreposições -- um problema inerente
à técnica de visualização utilizada --, como conseguimos melhorar o aspecto
visual das arestas nos gráficos.

\subsection{Detecção de outliers}
Os gráficos gerados representam comunidades de indivíduos. O posicionamento
de pessoas de comunidades diferentes, em seus respectivos gráficos, depende
diretamente da similaridade entre tais pessoas. No entando, existem pessoas
que não apresentam padrões de visitas similares aos de nenhuma das outras
pessoas.

O gráfico abaixo mostra a soma das distâncias de cada indivíduo em relação
a todos os outros. A partir do gráfico, é possível perceber que alguns
indivíduos possuem distâncias muito altas em relação aos outros, e,
consequentemente, têm hábitos não similares aos de qualquer outra pessoa.
\begin{figure}[!h]
\centering
\includegraphics[scale=.07]{../visualizacao-tp1/img/sum_distance.pdf}
\caption{Soma de distâncias de todas para todas as pessoas}
\end{figure}

\subsection{Filtragem de arestas}
Sobre o resultado inicial dos gráficos, notamos que os grafos de {\it donuts}
eram completos, ou seja, tinham arestas entre todos os indivíduos.
A fim de reduzir ainda mais a poluição visual, decidimos parametrizar
a inclusão de arestas nos gráficos. Assim, para que exista uma aresta
entre dois indivíduos, os mesmos têm que ter visitado determinado lugar
uma quantidade mínima de vezes -- definida na interface web desenvolvida.

\subsection{Heatmap}
Após a implementação da visualização por {\it heatmap}, podemos perceber
que as pessoas com mesmo identificador eram, no geral, agrupadas
como similares. Além disso, vimos também que esta abordagem produz resultados
visualmente mais claros e intuitivos que a original -- {\it donut graphs}.

\subsection{Métricas de similaridade}
Para o trabalho, foram implementadas duas métricas de similaridade.
A primeira, sugerida na especificação, foi a distância euclideana.
A segunda métrica utilizada foi a similaridade de cosseno.
Em relação aos resultados, ambas as métricas produziram resultados visuais
similares, apesar da métrica de similaridade de cosseno ser mais robusta
sobre dados com alta dimensionalidade.

% -----------------------------------------------------------------------------
% Conclusão
\section{Conclusão}
Durante a análise visual de dados, o usuário é o responsável
por identificar, distinguir, categorizar e comparar informações a ele
apresentadas. Desta forma, uma boa estratégia de visualização é aquela
que facilita o trablho do usuário, ou seja, que propicia a identificação
de padrões com pouco esforço.

Com isso, notamos durante o trabalho, que a técnica de {\it Heatmap},
relativamente mais simples que os {\it Donut graphs}, tem resultado visual
mais representativo. Ou seja, é possível identificar, de forma mais intuitiva,
os padrões existentes entre as comunidades.

Foram implementadas duas interfaces web, uma para cada visualização
desenvolvida. Para acessar o trabalho principal, feito com os gráficos
de {\it donuts}, o link utilizado é 
\url{http://dcc.ufmg.br/~periclesrafael/tp-visu/visualization.html}.
E, para o trabalho extra, o mesmo foi hospedado em
\url{http://dcc.ufmg.br/~rubens/dv/heatmap.html}.

\end{document}
