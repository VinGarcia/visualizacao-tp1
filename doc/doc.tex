% -----------------------------------------------------------------------------
% Document type
\documentclass[12pt]{article}

% -----------------------------------------------------------------------------
% Packages
\usepackage{amsmath}
\usepackage{amssymb}
\usepackage{graphicx}
\usepackage{sbc-template}
\usepackage[utf8]{inputenc}
\usepackage[pdftex]{hyperref}
\usepackage[brazilian]{babel}
\usepackage[lined,ruled,linesnumbered]{algorithm2e}

% -----------------------------------------------------------------------------
% Definitions
\title{Trabalho Prático 1 \\ Visualização de Padrões em Comunidades}
\author{Larissa Leijôto, Péricles Alves, Rubens Emilio, Vinícius Garcia}
\address{Departamento de Ciência da Computação \\
	Universidade Federal de Minas Gerais
	\email{$[$larissa,periclesroalves,rubens,vinicius$]$@dcc.ufmg.br}
}

\begin{document}
\hypersetup{pdfborder = {0 0 0}}
\maketitle

% -----------------------------------------------------------------------------
% Introdução
\section{Introdução}
Visualizar informação não é uma tarefa simples: além dos problemas inerentes
ao processamento dos dados, construir visualizações claras e objetivas requer
conhecimentos básicos sobre o processo cognitivo.
Deixando de lado as etapas de coleta e tratamento da base de dados --
que são, em si, problemas complexos --, existem diversas estratégias para
extrair informações relevantes de conjuntos de dados.

Diversos algoritmos de áreas como mineração de dados, aprendizado de máquina
e computação natural, processam os dados a fim de identificar padrões
implícitos. O volume dos dados e a existência de ruídos são exemplos
de problemas recorrentes nestas áreas de pesquisa. Ao final do processamento,
é necessário exibir as informações de forma objetiva, valorizando os resultados
obtidos, e permitindo novas conclusões sobre os dados.

A tarefa de visualizar dados, além de lidar com os desafios discutidos
acima, ainda depende de conhecimentos acerca do processo cognitivo.
É preciso ter um mínimo de noção sobre como o ser humano percebe e interpreta
informações visuaos. Só então é possível decidir como exibir resultados
pré-computados, ou ainda, como exibir os dados iniciais para que o leitor
consiga, com relativamente pouco esforço, tirar conclusões valiosas.

Neste trabalho, será implementada uma técnica de visualização de padrões
em grafos. A visualização será feita sobre conjuntos disjuntos de pessoas
(comunidades), que compartilham um conjunto predefinido de lugares
que costumam visitar. O resultado do trabalho é uma representação gráfica
para cada conjunto de pessoas. Assim, espera-se que, ao analisar mais de uma
comunidade, seja possível identificar padrões visuais entre as mesmas.

% -----------------------------------------------------------------------------
% Padrões em Comunidades 
\section{Padrões em Comunidades}
A base de dados a ser processada contém dados ...

\subsection{Donut Graphs}
Para representar cada indivíduo ...

\subsection{Posicionamento e Arestas}
O posicionamento de cada pessoa em sua respectiva comunidade indica ...
Sempre que pessoas de uma mesma comunidade visitam um mesmo lugar, é desenhada
uma aresta entre as mesmas, possibilitando a percepção de mais padrões visuais.

% No processo de análise visual de dados, há algumas tarefas analíticas comumente executadas pelo usuário tais como identificar, localizar, distinguir, categorizar, agrupar, distribuir, ordenar, comparar, associar, correlacionar, entre outras. O projeto de visualizações de dados deve, então, levar em conta essas necessidades básicas e facilitar essas tarefas analíticas. Nesse trabalho prático, vocês devem implementar um algoritmo que facilite ao usuário / leitor a identificação e associação de objetos similares visualmente, ou seja, através de atributos visuais pré-atentivos.

% -----------------------------------------------------------------------------
% Detalhes de Implementação
\section{Detalhes de Implementação}
Detalhes de implementação...

% -----------------------------------------------------------------------------
% Padrões Encontrados
\section{Padrões Encontrados}
Vimos os repousantes externos...

E, como um todo, é possível perceber grande similaridade entre
as comunidades...

% -----------------------------------------------------------------------------
% Conclusão
\section{Conclusão}
Conclusão...

% -----------------------------------------------------------------------------
% Referências
% \bibliographystyle{sbc}
% \bibliography{ref}

\end{document}
